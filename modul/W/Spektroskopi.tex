\praktikum[W3]{Spektrum Cahaya Lampu}
\subsection{Pendahuluan}

\begin{wrapfigure}{r}{0.4\textwidth} % Menempatkan gambar di sebelah kanan (r) dengan lebar 40% dari lebar teks
  \centering
  \includegraphics[width=0.4\textwidth]{Gambar/Spektroskop_animasi (1).png}
\end{wrapfigure}
    Pernahkah anda mengenal grub band legendaris bernama “Pink Floyd”? Jika anda mengenal grub band tersebut anda tidak akan asing dengan warna pelangi yang menjadi logo dari band tersebut. Dalam logo tersebut seakan-akan merekontruksi percobaan jenius Newton yang melewatkan cahaya putih pada sebuah prisma dan menjadi pelangi, sungguh menarik. Hal itu berarti bahwa setiap cahaya putih tersusun dari berbagai panjang gelombang. Umumnya, fenomena pelangi terbentuk akibat dispersi cahaya matahari yang melewati lensa berupa butiran air. Akan tetapi, bagaimana jika kita membentuk pola pelangi tersebut berdasarkan difraksi? 
	Dalam percobaan ini, dua artis utama yang akan tampil adalah si klasik pijar dan si kece fluoresense. Keduanya sama-sama mempunyai kekuatan untuk memancarkan cahaya, tetapi keduanya sama sekali memiliki kepribadian yang berbeda. Lampu pijar, si penggila panas menahan elektron yang melewatinya dan memaksa dirinya menjadi matahari kecil. Cahaya yang dipancarkanya mencakup seluruh spektrum cahaya tampak secara kontinu. Sedangkan lampu fluoresense, dengan memanfaatkan kekuatan kuantum, melesatkan elektron yang menabrak merkuri yang bisa menghasilkan sinar UV. Tapi tunggu, UV bukanlah cahaya yang bisa diamati mata manusia, tetapi fosfor lah yang menjadi konverter UV menjadi cahaya tampak. Tetapi spektrum yang dihasilkan tidak mulus seperti pijar, melainkan terkotak-kotak seperti rasi bintang di langit. 



\subsection{Tujuan}
Tujuan dari praktikum ini adalah:
\begin{enumerate}
    \item Memahami dispersi cahaya akibat kisi difraksi. 
    \item Memahami pembentukan cahaya tampak dari lampu pijar.
    \item Mengidentifikasi karakteristik spektrum lampu pijar dan lampu fluoresense.
    \item mengidentifikasi karakteristik spektrum emisi gas, dan mengidentifikasi material dari emisi spektral.

\end{enumerate}
\subsection{Alat dan Bahan}
Peralatan yang digunakan praktikum ini adalah :
\begin{table}[h!]
\centering
\begin{tabular}{ccc}
\toprule
\hline
\textbf{Nama Alat} & \textbf{Jumlah} & \textbf{Detail} \\
\hline
Platform Eksperiment & 1 & - \\
Skala Pengukur & 1 &  untuk mengukur jarak spektrum. \\
Lampu Pijar & 1 & - \\
Lampu Fluoresen & 1 & - \\
Kisi Difraksi (500 lines/mm) & 1 & - \\
Pita Pengukur & 1 & untuk mengukur jarak  \\
Adaptor AC/DC & 1 & Sebagai sumber listrik \\
Kabel Daya & 1 & - \\
\hline
\bottomrule
\end{tabular}
\caption{Daftar alat untuk praktikum Spektrum Cahaya}
\end{table}
\begin{figure}[H]
    \centering
    \includegraphics[width=0.5\linewidth]{Gambar/Spektrum_cahaya.png}
    \caption{Skema alat praktikum Spektrum cahaya}
    \label{fig:enter-label}
\end{figure}
\subsection{Pre-Lab}
\begin{enumerate}
    \item Jelaskan mengenai cahaya  polikromatik dan cahaya monokromatik!
    \item Apa itu dispersi cahaya dan bagaimana hubungan cahaya dengan garis spektral yang dihasilkan dari fenomena dispersi cahaya?
    \item Jelaskan mengenai cahaya tampak dan nilai panjang gelombang dari masing-masing cahaya tampak tersebut!
    \item Bagaimana lampu pijar dan lampu fluoresen menghasilkan cahaya tampak ketika diberikan tegangan?
    \item Jelaskan apa itu lampu pijar dan lampu fluoresen dan bagaimana kaitannya dengan spektrum cahaya yang dihasilkan?
    \item Bagaimana suhu sampel mempengaruhi spektrum emisinya?
    \item Apa pentingnya garis spektral dalam spektrum emisi, dan bagaimana garis-garis tersebut terkait dengan transisi energi?
    \item Jelaskan parameter apa  saja yang menyebabkan pelebaran/pemisahan garis spektrum pada cahaya lampu diatas!
    \item carilah gambar mengenai spektrum warna kontinu dan spektrum warna diskrit!
\end{enumerate}
\subsection{In-Lab}
\subsubsection*{Spektrum Kontinu dari Lampu Pijar}
\begin{enumerate}
    \item Pasang peralatan percobaan sesuai diagram.
    \item Pasang lampu pijar pada dudukan geser menggunakan batang baja tahan karat, lalu hubungkan ke adaptor sumber listrik.
    \item Tempatkan kisi difraksi pada dudukan geser dan pastikan sejajar dengan sumber cahaya.
    \item Nyalakan lampu pijar.
    \item Amati spektrum yang dihasilkan melalui kisi difraksi.
    \item Ukur jarak spektrum warna dari sumber cahaya dengan menggunakan skala pengukur.
    \item Ukur jarak vertikal antara kisi difraksi dan sumber cahaya.
    \item Catat hasil pengukuran dan bandingkan dengan panjang gelombang teoretis.
\end{enumerate}

\subsubsection*{Spektrum Emisi dari Lampu Fluoresen}
\begin{enumerate}
    \item Pasang peralatan percobaan sesuai diagram.
    \item Pasang lampu fluoresen pada dudukan geser menggunakan batang baja tahan karat, lalu hubungkan ke sumber listrik.
    \item Tempatkan kisi difraksi pada dudukan geser dan pastikan sejajar dengan sumber cahaya.
    \item Nyalakan lampu fluoresen.
    \item Amati spektrum yang dihasilkan melalui kisi difraksi.
    \item Ukur jarak spektrum warna dari sumber cahaya dengan menggunakan skala pengukur.
    \item Ukur jarak vertikal antara kisi difraksi dan sumber cahaya.
    \item Gunakan persamaan kisi difraksi untuk menghitung panjang gelombang.
    \item Catat hasil pengukuran dan bandingkan dengan panjang gelombang teoretis spektrum emisi merkuri.
\end{enumerate}

\begin{table}[h]
    \centering
    \begin{tabular}{lccc}
        \toprule
        \midrule
        \textbf{Warna} & \textbf{Panjang Gelombang(nm)} \\
        \midrule
        Ungu   & 402.8  \\
        Biru   & 447.3  \\
        Cyan   & 483.1  \\
        Hijau  & 537.9   \\
        Kuning & 568.6  \\
        Oranye & 606.4  \\
        Merah  & 639.8  \\
        \midrule
        \bottomrule
    \end{tabular}
    \caption{Panjang Gelombang Cahaya dari Lampu Pijar dan Lampu Fluoresen}
    \label{tab:spektrum}
\end{table}

\begin{table}[h]
    \centering
    \begin{tabular}{cccc}
        \toprule
        \midrule
        \textbf{Warna} & \textbf{Panjang Gelombang (nm)}\\
        \midrule
        Biru   & 435.2   \\
        Hijau  & 545.6   \\
        Kuning & 583.8   \\
        Merah  & 621.3   \\
        \midrule
        \bottomrule
    \end{tabular}
    \caption{Panjang Gelombang Cahaya dari Lampu Pijar dan Lampu Fluoresen}
    \label{tab:spektrum}
\end{table}

\subsection{Analisis Data}
\begin{enumerate}
    \item setelah mendapatkan data hasil pengukuran, ubah data pengukuran ke dalam satuan nm.
    \item hitung nilai panjang gelombang pada masing - masing warna dengan menggunakan rumus dibawah ini,
    \begin{equation}
        \lambda = \frac{d \cdot x}{\sqrt{l^2 + x^2}}
    \end{equation}
    dengan:
    \begin{itemize}
        \item $d$ = lebar kisi (500 lines/mm)
        \item $l$ = jarak dari pusat terang pertama ke pita terang tertentu
        \item $x$ = jarak dari kisi ke sumber cahaya
    \end{itemize}
    dimana delta d merupakan jarak diantara slit, s merupakan jarak sumber cahaya ke kisi difraksi, dan l merupakan jarak diantara sumber cahaya dengan spektrum warna.
    \item kemudian hitung nilai rata - rata pada masing - masing panjang gelombang.
    \item selanjutnya bandingkan nilai panjang gelombang hasil eksperimen dengan panjang gelombang secara teori
    \item hitung nilai error dari panjang gelombang yang telah diperoleh.
    pengolahan data dapat menggunakan software excel, python, MATLAB dan sebagainya.
\end{enumerate}
\subsection{Post-Lab}
\begin{enumerate}
    \item Jelaskan bagaimana garis spektral dapat terbentuk saat lampu diamati melalui kisi difraksi!
    \item Apakah Garis spektral yang dihasilkan oleh lampu pijar dan lampu fluoresen berbeda atau sama? dan jelaskan alasannya!
    \item Jika sumber cahaya digantikan dengan lampu gas seperti helium apakah garis spektral yang dihasilkan akan berbeda? jelaskan alasannya?
    \item Bagaimana hubungan antara jarak kisi dengan sumber cahaya pada pengamatan spektrum warna? apa yang terjadi pada hasil spektrum warna tersebut?
    \item Jelaskan data yang kamu peroleh. Apakah data tersebut sudah dapat dikatakan akurat atau belum? dibandingkan dengan nilai teori yang sudah ditentukan sebelumnya
    \item Apa saja kesalahan eksperimental yang mungkin anda temui dalam percobaan ini? Bagaimana anda dapat menghindarinya?
\end{enumerate}
