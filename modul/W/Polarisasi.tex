\praktikum[W2]{Determinasi Sudut Brewster}
\subsection{Pendahuluan}

\begin{wrapfigure}{r}{0.4\textwidth} % Menempatkan gambar di sebelah kanan (r) dengan lebar 40% dari lebar teks
  \centering
  \includegraphics[width=0.4\textwidth]{Gambar/Jendela_Brewster.png}
\end{wrapfigure}
Polarisasi adalah trik canggih yang dimainkan cahaya, di mana ia tiba-tiba memutuskan untuk hanya bergerak dalam satu arah saja, seolah-olah ia akhirnya memutuskan untuk berhenti dan mulai fokus pada suatu titik. Biasanya, cahaya itu bebas dan liar, bergetar di segala arah yang tegak lurus terhadap arah rambatnya. Tapi dengan polarisasi, cahaya memutuskan untuk "menyatu" dan hanya bergerak dalam satu garis lurus. Cara paling sederhana untuk membuatnya rapi adalah dengan memaksanya memantul dari permukaan reflektif. Pada titik ini, kita masuk ke dalam dunia yang lebih menarik: sudut Brewster. Hal ini seakan-akan menjadi pertemuan dramatik antara cahaya dan permukaan reflektif. Ketika cahaya datang dengan sudut yang tepat, yaitu sudut Brewster, ia sepenuhnya berubah, terpolarisasi, dan berjanji untuk hanya berosilasi dalam satu arah—seperti saat sebuah aliran sungai akhirnya menemukan jalannya yang tenang, bebas dari batuan dan halangan lainnya. Pada sudut ini, cahaya yang dipantulkan hanya meninggalkan satu arah getaran, sementara arah lainnya dengan elegan menghilang ke dalam medium lain. Semua ini bisa dihitung dengan rumus Snell, tentu saja, yang akan menunjukkan kepada kita berapa derajat agar cahaya bisa mendapatkan ‘terapi polarisasi’ terbaiknya.


\subsection{Tujuan}
Tujuan dari Praktikum ini adalah 
\begin{enumerate}
    \item Memahami konsep sudut brewster.
    \item Menganalisis Polarisasi Cahaya yang Dipantulkan dan Ditransmisikan.
    \item Menentukan besarnya sudut Brewster dan indeks bias glass side dengan menggunakan metode pengukuran refleksi cahaya.
\end{enumerate}

\subsection{Alat dan Bahan}
Peralata yang digunakan praktikum ini adalah :
\begin{table}[H]
\centering
\begin{tabular}{ccc}
\toprule
\midrule
\textbf{Nama Alat} & \textbf{Jumlah} & \textbf{Detail} \\
\hline
Set Alat Praktikum Polarisasi & 1 & - \\
Kaca Non-Transmisif & 1 & Digunakan untuk refleksi cahaya \\
Polarisator dengan Busur Derajat & 1 & Umengontrol polarisasi cahaya \\
Laser merah (650 nm) & 1 & - \\
Adaptor & 1 & Sumber daya untuk laser \\
Fotodetektor & 1 & - \\
Amplifier & 1 & Menguatkan sinyal dari fotodetektor \\
Multimeter & 1 & Mengukur tegangan dari fotodetektor \\
Kabel Patch & 2 & Menghubungkan perangkat elektronik \\
Kabel Daya & 1 & Menyediakan daya ke laser \\
\midrule
\bottomrule
\end{tabular}
\caption{Daftar Alat Determinasi Sudut Brewste}
\end{table}
\subsection{Pre-Lab}
\begin{enumerate}
    \item Sebutkan dan jelaskan macam macam jenis polarisasi  berdasarkan arah getaran gelombang cahaya dan hubungan fase antara dua komponen gelombang cahayanya
    \item Cahaya alami (misalnya cahaya matahari) terdiri dari gelombang elektromagnetik yang memiliki arah getaran yang acak. Jelaskan bagaimana sebuah polarizer dapat memfilter cahaya alami dan hanya memungkinkan gelombang dengan arah getaran tertentu untuk melewati.
    \item Jelaskan secara konsep bagaimana cahaya terpolarisasi ketika dipantulkan dari permukaan air atau kaca. Apa yang terjadi pada cahaya yang dipantulkan dan bagaimana polarisasinya terkait dengan sudut datang?
    \item Apa yang dimaksud dengan sudut Brewster? dan polarisasi apa yang terjadi pada refleksi cahaya?
    \item Mengapa sudut Brewster hanya berlaku untuk cahaya yang dipantulkan dan bukan untuk cahaya yang diteruskan (refraksi)?
    \item Apa yang terjadi pada intensitas cahaya yang dipantulkan pada sudut Brewster? Mengapa hal ini terjadi?
    \item Apakah sudut Brewster dipengaruhi oleh panjang gelombang cahaya? Jelaskan bagaimana pengaruhnya.
\end{enumerate}
\subsection{In-Lab}
\subsubsection*{Instalasi Alat}  
\begin{enumerate}
    \item Pastikan cahaya laser sudah disesuaikan dengan benar sehingga mengenai slide kaca.
    \item Potongan kaca reflektif harus diposisikan secara vertikal atau tegak lurus dengan panggung proyektor.
    \item Laser yang datang harus mengenai permukaan kaca dan dipantulkan ke detektor foton.
\end{enumerate}
    \begin{figure}[htbp]
        \centering
        \includegraphics[width=0.5\linewidth]{Gambar/Alat2_polarisasi.png}
        \label{fig:enter-label}
    \end{figure}
    \infobox{Tempatkan alat diatas di ruangan gelap untuk menghindari pencahayaan luar}
    \Step{Busur lingkaran terintegrasi dengan dua dudukan dimana bagian luar memilikan putaran kenaikan sebesar $2^\circ$ dan bagian dalam sebesar $0.1^\circ$.Untuk mengukur derajat pada busur: \\ Pertama, putar bagian dalam $0^\circ$ diantara posisi dua sudut terdekat.
    Lihat tanda bagian dalam yang berimpit dengan bagian luar dan kalikan nilainya dengan $0.1^\circ$ dan tambahkan dengan nilai garis terendah bagian luar sebelum $0^\circ$ bagian luar.\\ 
    Contoh pembacaan ditunjukkan pada gambar disamping}
    \begin{figure}[H]
        \centering
        \includegraphics[width=0.7\linewidth]{Gambar/Alat_Skema_Polarisasi.png}
        \caption{Skema Alat Sudut Brewster}
        \label{fig:enter-label}
    \end{figure}
\subsubsection*{Prosedur Praktikum}
\begin{enumerate}
    \item Letakkan sisi kaca menghadap laser dan pastikan ketika busur lingkaran diletakkan pada sudut $0^\circ$, sinar laser direfleksikan kembali ke sumber laser.
    \item Temukan arah sumbu polarisasi yang sejajar dengan sisi kaca: Ketika sudut refleksi, sudut Brewster menunjukkan bahwa sinar sejajar akan semuanya ditransmisikan. Putar sisi kaca ke sudut dan mulai putar polarisator sampai didapatkan nilai tegangan minimum.
    \item Putar kembali sisi kaca ke sudut $20^\circ$ dan putar hingga $80^\circ$ dengan kenaikan perubahan sudut sebesar $10^\circ$ per langkah dan ukur intensitas cahaya. Pada sudut $50^\circ$-$60^\circ$, kenaikan per langkahnya menjadi $2^\circ$, dan pada $54^\circ$-$56^\circ$ menjadi $0.2^\circ$, karena pada daerah ini sudut Brewster teramati.\\
    \infobox{ Penting untuk memastikan bahwa lengan rotasi dari penyangga fotodetektor cukup longgar agar dapat bergerak dengan bebas, namun tidak terlalu longgar sehingga akan tergelincir turun sepanjang penyangga. Jika terlalu kencang, maka posisi penyangga akan ikut berputar bersama fotodetektor. Hal ini akan menyebabkan sudut awal pengukuran Anda bergeser dan mempengaruhi hasil eksperimen Anda.}
    \Step{
    Saat memutar sisi kaca, putar juga knop pada bagian laser agar sinar laser benar benar tepat mengenai permukaan fotodetektor
    }
    \item Pantulan cahaya dideteksi oleh fotodetektor pada lengan rotasi yang terkoneksi pada penyangga. Pastikan fotodetektor diletakkan tepat pada pantulan cahaya sehingga intensitas maksimum teramati.
    \item Catat tegangan yang terbaca pada multimeter. Setelah mencatat intensitas cahaya terpolarisasi sejajar, putar polarisator sebesar $90^\circ$ untuk percobaan dengan cahaya yang terpolarisasi tegak lurus.
    \item Ulangi langkah 3 hingga 5 untuk pengukuran cahaya terpolarisasi tegak lurus.
    \item Isikan hasil praktikum pada Tabel 1.
    \item Buatlah grafik intensitas cahaya terpolarisasi sejajar dan tegak lurus versus sudut pantul.
    \end{enumerate}
\subsection{Analisis Data}
\begin{enumerate}
    \item plot hasil putaran sudut refleksi dengan tegangan yang terbaca pada multimeter
    \item kemudian cari nilai minimum pada kurva hasil polarisasi sejajar dan dapatkan nilai sudut refleksinya
    \item Gunakan persamaan 
    \begin{equation}
        \theta = \tan^{-1}{(\frac{n_1}{n_2})}
    \end{equation}
    untuk mencari nilai indeks bias kaca $n_1$
\end{enumerate}
\subsection{Post-Lab}
\begin{enumerate}
    \item Mengapa dalam polarisasi sejajar tidak ada cahaya yang dipantulkan?
    \item Apa indikasi utama dalam data hasil praktikum yang menunjukkan bahwa sudut tertentu adalah sudut Brewster?
    \item Mengapa pengukuran intensitas refleksi untuk polarisasi sejajar harus dilakukan dengan resolusi sudut yang lebih kecil di sekitar sudut Brewster?
    \item Apa saja faktor yang dapat menyebabkan hasil eksperimen berbeda dari nilai teoritis sudut Brewster?
    \item Bagaimana pengaruh ketidakakuratan dalam penyetelan sudut datang terhadap hasil eksperimen?
    
\end{enumerate}