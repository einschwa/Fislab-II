\praktikum[W1]{Difraksi Celah Tunggal}
\subsection{Pendahuluan}
\begin{wrapfigure}{r}{0.35\textwidth}
  \centering
  \includegraphics[width=0.4\textwidth]{Gambar/Difraksi (1).png} 
\end{wrapfigure}
Pernahkah anda melihat sebuah konser dari grub band yang sangat terkenal? Tentu saja, katakanlah Dewa 19 show bersama Slank dan Gigi pasti akan dipenuhi oleh lautan manusia yang menonton konser musik tersebut. Bayangkan saja ketika konser selesai, lautan manusia tersebut akan keluar stadion melalui 1 pintu. Apa yang terjadi? Anda akan mengamati lautan manusia tersebut menyebar seperti mengalami pelenturan. Hal yang sama bisa terjadi pada gelombang yang melewati sebuah celah sempit. Dan yap! Cahaya mengembang, melentur, dan difraksi terjadi. Eureka!!
Pada tahun 1678,  seorang Fisikawan Berkebangsaan Belanda bernama Christiaan Huygens membayangkan setiap titik gelombang yang melewati celah sempit membentuk muka gelombang baru yang meluncur ke segala arah. arah,gelombang ini berjalan dengan kecepatan yang sama dengan gelombang sumbernya, jadi mereka bersaing untuk sampai ke depan, dan pada akhirnya muka gelombang membentuk ‘pabrik tak terbatas’ yang terus memproduksi gelombang baru tanpa henti dan gelombang kecil ini saling bertemu dan membentuk muka gelombang yang baru. Tentu saja ini adalah terobosan ide yang sangat cemerlang untuk menjelaskan perilaku aneh gelombang. Dalam praktikum ini, kita akan lebih banyak membahas difraksi Franhoufer yaitu ketika lebar celah jauh lebih kecil dibandingkan dengan jarak celah ke layar, sehingga cahaya yang nampak akan sejajar. 

\subsection{Tujuan}
Tujuan dari praktikum ini adalah:
\begin{enumerate}
    \item Memahami konsep difraksi cahaya khususnya pada celah tunggal.
    \item Menentukan panjang gelombang dari laser.
    \item Memahami pengaruh lebar celah dan jarak antara layar dengan celah terhadap pola difraksi. 
    \item Menentukan pola intensitas cahaya pada difraksi celah tunggal.
\end{enumerate}
\subsection{Alat dan Bahan}
Peralatan yang digunakan praktikum ini adalah :
\begin{table}[htbp]
    \centering
    \begin{tabular}{ccc}
        \toprule
        \midrule
        \textbf{Nama Alat} & \textbf{Jumlah} & \textbf{Detail} \\
        \hline
        Aluminum platform & 2 & - \\
        Layar with mikrometer & 1 & - \\
        Laser hijau($\lambda = 520$ nm) & 1 & - \\
        Laser merah($\lambda = 650$ nm) & 1 & - \\
        Adaptor & 1 & - \\
        Kisi celah tunggal & 1 & $a = 0.05, 0.08, 0.11, 0.14$ mm \\
        set alat Photodetector & 1 & Detector, Signal amplifier, Multimeter \\
        Penggaris & 1 & - \\
        Power kabel & 1 & - \\
        \midrule
        \bottomrule
    \end{tabular}
    \caption{Daftar alat untuk percobaan Difraksi Celah tunggal}
    \label{tab:alat_single_slit}
\end{table}
\begin{figure}[H]
    \centering
    \includegraphics[width=1.0\linewidth]{skemadifraksi.png}
    \caption{Skema alat praktikum Difraksi celah tunggal}
    \label{fig:enter-label}
\end{figure}
\subsection{Pre-Lab}
\begin{enumerate}
    \item Jelaskan yang dimaksud peristiwa difraksi Cahaya dan apa saja syarat agar terjadi peristiwa difraksi Cahaya?
    \item Kemudian jelaskan peristiwa difraksi diatas menggunakan prinsip huygens?
    \item Dari persamaan kisi difraksi, Jelaskan pengaruh lebar celah terhadap pola difraksi ?
    \item Jelaskan pengaruh Panjang helombang terhadap pola difraksi ?
    \item Bagaimana hubungan antara lebar celah dan pola difraksi yang terbentuk pada layer?
    \item Jelaskan pengaruh jarak celah ke layer terhadap pola difraksi ?
    \item Mengapa Cahaya mengalami difraksi saat mengalami celah sempit?
\end{enumerate}
\subsection{In-Lab}
\subsubsection*{Instalasi Alat}
\begin{enumerate}
    \item Susun peralatan seperti yang diilustrasikan pada Gambar.4.
    \item Hubungkan amplifier (10.2.) ke fotodetektor (10.1.) dan juga ke multimeter (10.3.) melalui kabel patch (10.4.). 
    \item Pasang slide dengan layar (2.) pada platform Aluminium (1.2.) dan pasang fotodetektor (10.1.) pada sisi belakang layar. 
    \item Pasang sumber Laser (8.1. atau 8.2.) pada dudukan slide (3.) dan hubungkan menggunakan Adaptor (8.3.) melalui kabel (8.4.) lalu pasang pada platform Aluminium pertama (1.1.).Gunakan dudukan magnet (4.) untuk mengamankan lubang (9.1.) di depan laser dan sesuaikan posisinya hingga laser langsung mengarah ke layar.
    \item Pastikan laser langsung mengarah ke fotodetektor yang menembus bagian tengah layar, sehingga multimeter menunjukkan pembacaan tegangan tertinggi dan
\end{enumerate}
\infobox{Tempatkan alat diatas di ruangan gelap untuk menghindari pencahayaan luar}
\warningbox{TINDAKAN PENCEGAHAN:
JANGAN PERNAH memaparkan sinar laser ke mata Anda dan orang lain.
Sebelum melakukan percobaan, lepaskan benda-benda yang memantulkan cahaya dari tubuh Anda seperti jam tangan, kalung, cincin, dll.
}
\subsubsection*{Penentuan panjang gelombang}
\begin{enumerate}
    \item Atur jarak antara layar dan celah sampai pada jarak yang sesuai dan catat jaraknya.\\
    \infowithimage{Setelah menyesuaikan jarak, Anda harus mengamati setidaknya tiga puncak (Dari A ke B) saat Anda memasang celah tunggal 0,05.}{Gambar/Difraksi_Panjang_Gel.png}
    \item Putar penyetel dudukan celah dan bahas hubungan antara lebar celah dan pola yang ditampilkan di layar. Kemudian pilih satu lebar celah dan setel hingga Anda melihat pola paling terang di layar. Pastikan pola tidak miring ke arah mana pun. 
    \item Putar roda pada dudukan laser ke kiri/kanan dan atas/bawah untuk menyesuaikan posisi laser hingga pola berada di tengah layar karena pinggiran terang di tengah sejajar tepat pada lubang.\\
    
    \tipswithimage{Untuk menggerakkan layar selama pengukuran, putar roda putih di sisi layar. Rotasi penuh akan menggeser layar sejauh 1 mm. Oleh karena itu, Anda dapat memilih kenaikan sebesar 1 mm atau 0,5 mm.}{Gambar/Difraksi_Step2.png}
    \item Tuliskan hasil pembacaan tegangan pada tabel dengan cara mengganti posisi secara bertahap dari satu sisi layar ke sisi lainnya (40 mm) dan gunakan software untuk membuat plot kurva cahaya.
    \infobox{Karena kita tidak dapat mengukur intensitas cahaya secara langsung, kita mengubahnya menjadi energi listrik di detektor dan mengukur voltasenya dengan multimeter digital. Oleh karena itu, voltase mewakili intensitas cahaya yang menyinari.}
    \item Gunakan Persamaan difraksi celah tunggal (1) untuk menghitung panjang gelombang eksperimen dan kesalahannya.
    
\end{enumerate}
\subsubsection*{Pengaruh lebar celah pada pola}
\Step{Sebelum merakit, tempelkan kertas tebal pada layar sehingga Anda dapat menggambar garis luar pola yang dihasilkan. Gunakan pensil kayu untuk menggambar garis. Jika tidak, kepala pena akan memantulkan laser saat Anda menggambar dan dapat menyebabkan kerusakan pada mata Anda.}
\begin{enumerate}
    \item Tentukan warna laser Anda, lakukan percobaan ini 
    \item Sesuaikan platform (1.2) karena celah dan layar memiliki jarak hingga pinggiran pada layar berada dalam rentang yang sesuai.
    \item Putar penyetel penahan celah (4.) hingga Anda memperoleh pola paling terang pada layar. Amati dan gambar garis luar pola yang dihasilkan pada layar yang dibuat dengan lebar celah masing-masing. 
    \item Ulangi dari (1) hingga (3) dengan tiga celah tunggal yang berbeda. 
    \item Matikan laser segera setelah selesai menggambar. 
    \item Bandingkan pola yang dibuat dengan lebar celah yang berbeda dan diskusikan.
\end{enumerate}
\infobox{ Saat membuat perbandingan, harap sejajarkan bagian tengah pola pada posisi yang sama sehingga Anda dapat membedakan perbedaannya dengan lebih mudah.}
\subsubsection*{Pengaruh jarak pada pola (dengan lebar celah dan panjang gelombang tetap)}
\begin{enumerate}
    \item Ganti kertas dengan yang baru dan tentukan warna laser yang Anda inginkan. 
    \item Putar penyetel penahan celah hingga Anda memperoleh pola paling terang di layar dan biarkan tetap di sana. 
    \item Atur platform aluminium kedua saat celah dan layar memiliki jarak hingga pinggiran pada layar berada dalam rentang yang sesuai. 
    \item Amati dan gambar garis luar pola yang dihasilkan pada layar yang dibuat dengan jarak masing-masing. 
    \item Ulangi dari (1) hingga (4) dengan tiga jarak yang berbeda.
    \item Matikan laser segera setelah selesai menggambar. 
    \item Bandingkan dan diskusikan polanya.
\end{enumerate}
\subsubsection*{Pengukuran Intensitas Cahaya Laser dengan Fotodetektor}
\begin{enumerate}
    \item Pasang dan Gunakan fotodetektor untuk mengukur intensitas cahaya di berbagai posisi di sepanjang layar.
    \item Hubungkan fotodetektor ke amplifier dan multimeter untuk mencatat tegangan keluaran.
    \Step{Gerakkan fotodetektor secara bertahap dari satu sisi layar ke sisi lainnya dengan kenaikan langkah yang tetap (misalnya 0.5 mm atau 1 mm).}
    \infobox{tegangan yang terbaca pada multimeter untuk setiap posisi.
Ulangi pengukuran untuk berbagai lebar celah (misalnya 0.05 mm, 0.08 mm, 0.11 mm, dll.)}
\end{enumerate}
\subsection{Analisis Data}
\begin{enumerate}
    \item Hitung untuk posisi pita hitam pertama, kedua, ketiga, dst. 
    \item Kemudian hitung panjang gelombang Pada pola difraksi celah tunggal menggunakan
        \[
        a \sin \theta = m \lambda
        \]
        dengan:
        a = lebar celah (mm), $\lambda$ = panjang gelombang cahaya (nm), m = nomor urutan minimum \ (m = 1, 2, 3, $\dots$), $\theta$ = sudut difraksi. Karena sudut $\theta$ dianggap kecil, persamaan diatas menjadi
        \[
        \lambda = \frac{a y_m}{m D}
        \]
        Dengan $D$ adalah jarak kisi ke layar dan $y_m$ jarak ke pita terang difraksi.
    \item Setelah itu, plot kurva hasil tegangan terhadap perubahan skala mikrometer pada layar 
    \item Selanjutnya bandingkan nilai Panjang gelombang hasil eksperimen dengan Panjang gelombang secara teori 
    \item Hitung nilai error nisbi dari Panjang gelombang dan intesitas yang diperoleh.
    \item Sekarang plot hasil pembacaan tegangan terhadap posisi dan bandingkan dengan intensitas sebenarnya (dengan asumsi bahwa Tegangan terhadap intensitas adalah sebanding) yang diberikan oleh bentuk persamaan 
    \begin{equation}
        \frac{I(\theta )}{I_0}= (\frac{\sin{\beta}}{\beta})^2
    \end{equation}
    Dimana nilai $\beta=\frac{\pi a \sin{\theta}}{\lambda}\approx \frac{\pi a x}{\lambda L}$ dimana $a$ adalah lebar kisi celah, $L$ adalah jarak kisi ke layar, dan $x$ adalah posisi.
    \item kemudian hitung error nisbi dari hasil perbandingan intensitas diatas.
\end{enumerate}
\subsection{Post-Lab}
\begin{enumerate}
    \item Bagaimana cara kerja kisi difraksi? Apa yang terjadi jika digunakan kisi difraksi dengan jumlah celah per satuan panjang yang lebih sedikit?
    \item Bagaimana jarak antara kisi difraksi ke layar memengaruhi pola interferensi yang diamati?
    \item Apa yang terjadi jika lebar celah diperbesar dalam difraksi celah tunggal?
    \item Berdasarkan hasil percobaan, faktor apa yang paling berpengaruh terhadap pola difrakssi Cahaya dan bagaimana hubungan antara panjang gelombang cahaya, lebar celah, dan interferensi?
    \item Jelaskan data yang kamu peroleh, apakah ada perbedaan signifikan antara nilai Panjang gelombang yang diperoleh secara teoritis? Jika ada, apa penyebabnya 
\end{enumerate}
