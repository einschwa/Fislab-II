\newpage
\praktikum[MP2]{Tetes Minyak Millikan}
\subsection{Pendahuluan}

\begin{wrapfigure}{r}{0.35\textwidth}
  \centering
  \includegraphics[width=0.4\textwidth]{Gambar/freepik__a-childrens-animated-drawing-with-black-contour-li__40141 (2).jpg} 
\end{wrapfigure}
Bayangkan kamu punya kue cokelat mini yang sangat ringan, hampir tak terlihat oleh mata telanjang. Tantangannya adalah: bagaimana menimbang sesuatu yang begitu kecil tanpa alat yang biasa digunakan? Kita mungkin akan putus asa mencoba mengukur kue ini. Tetapi sekarang kita tidak membicarakan sebuah kue coklat, melainkan sebuah muatan elektron. Benda ini begitu kecil, bahkan mustahil kita lihat tanpa mikroskop super canggih

Namun, ada seorang ilmuwan Amerika yang begitu cerdik, Robert Andrews Millikan, Dia memikirkan trik jenius untuk mengukur muatan elektron ini. Seperti seorang alkemis kuno ia merakit alat misterius yang tampak seperti sesuatu dari fiksi ilmiah: sebuah perangkat dengan dua pelat logam besar dan tetes minyak kecil yang melayang di antara mereka. Dengan sedikit permainan listrik dan trik optik, Lalu, eureka! Ia menemukan cara untuk menjinakkan muatan elektron yang liar dan hampir tak terlihat ini.
Siapa sangka millikan berhasil menunjukkan bahwa muatan listrik tidak sembarangan; ia memiliki unit dasar yang disebut muatan elementer, atau e, yang nilainya sekitar \(1.602 \times 10^{-19}\) coulombs.



\subsection{Tujuan}
Tujuan dari praktikum ini adalah :
\begin{enumerate}
    \item Memahami kuantisasi pada muatan tetes minyak millikan.
    \item Merekontruksi dari percobaan millikan.
    \item Mencari muatan elementer pada tetes minyak millikan.
\end{enumerate}
\subsection{Alat dan Bahan}
Untuk alat dan bahan pada praktikum ini ditunjukkan pada tabel berikut

\begin{longtable}{ccc}
\toprule
\hline
\textbf{Nama Alat} & \textbf{Jumlah} & \textbf{Detail} \\
\hline
\endfirsthead % Bagian header tabel untuk halaman pertama

\hline
\textbf{Nama Alat} & \textbf{Jumlah} & \textbf{Detail} \\
\hline
\endhead % Bagian header tabel untuk halaman berikutnya

\hline
\bottomrule
\endfoot % Bagian footer tabel untuk halaman berikutnya
    \textit{Millikan’s oil drop apparatus} & 1 & Alat utama untuk eksperimen \\
    Plat Kapasitor & 1 & Memberikan medan listrik ke tetes minyak \\
    \textit{Voltage supply} & 1 & Sumber tegangan untuk plate capacitor \\
    Mikroskop & 1 & Untuk mengamati pergerakan tetes minyak \\
    Lampu & 1 & Sumber cahaya untuk eksperimen \\
    Penyemprot minyak & 1 & Untuk menghasilkan tetes minyak kecil \\
    Stopwatch & 1 & Mengukur waktu jatuh dan naik tetes minyak \\
\end{longtable}

\subsection{Pre-Lab}
\begin{enumerate}
    \item Millikan awalnya menggunakan air untuk eksperimennya. Mengapa dia kemudian memilih minyak sebagai gantinya?
    \item Bagaimana Millikan mengisolasi satu muatan elektron? Kita tahu bahwa dia bereksperimen dengan tetesan minyak,
    bukan elektron tunggal.
    \item Melalui Mekanika Hukum Newton, turunkan persamaan gerak dari tetes minyak millikan pada pada kecepatan terminal 
    $v_d$ jika diketahui besaran $E$, massa jenis droplet minyak $\rho$ ,massa jenis udara $\rho_u$ , viskositas udara $\eta$, percepatan gravitasi $g$!
    \item Dari pertanyaan diatas, manakah yang lebih mudah untuk diukur, massa dari tetes minyak ? atau jari jari tetes minyak ? Jelaskan pula metode matematis yang digunakan untuk mengetahui besaran tersebut.
    \item Dari metode diatas Kemudian jelaskan bagaimana Millikan melakukan eksperimennya. Bagaimana dia mengukur muatan minyak?
    \item Kemudian Bagaimana percobaan Millikan dapat membuktikan bahwa muatan listrik bersifat terkuantisasi dan selalu berupa kelipatan dari muatan dasar tertentu?
\end{enumerate}
\subsection{In-Lab}
\begin{enumerate}
    \item Siapkan minyak ke dalam gelas beker. Ambil menggunakan pipet dan masukkan minyak ke dalam atomizer. Pasang atomizer menuju tabung.
    \item Hubungkan kabel dan rangkai alat sesuai skema percobaan seperti yang ditunjukkan dalam Gambar di bawah.
    \begin{figure}[htbp]
        \centering
        \includegraphics[width=0.5\linewidth]{Gambar/Millikan.png}
        \caption{Skema Alat Percobaan Tetes Minyak Millikan}
        \label{fig:enter-label}
    \end{figure}
    \item Semprot minyak dengan memompa atomizer sebanyak 3-5 kali. Tinjau ke mikroskop, cek apakah butiran minyak dapat terlihat di dalam tabung.
    \item Gunakan counter untuk mengukur waktu yang ditempuh satu tetesan minyak untuk menempuh suatu jarak yang telah ditentukan, misalkan $d_1$.
    \item Selanjutnya, apabila satu butir tersebut telah mencapai batas bawah bidang pandang mikroskop, segera nyalakan tegangan masuk pada tabung tetes Millikan dengan power supply hingga terlihat tetesan minyak bergerak. Catat tegangan yang dibutuhkan satu butir minyak sepenuhnya diam. Tegangan ini merupakan \textit{balancing voltage}. 
    \item Selanjutnya, perbesar tegangan masuk hingga minyak bergerak ke atas, ukur waktu dan jarak tempuh hingga minyak mencapai batas atas bidang pandang mikroskop semisal $d_2$.
    \item Ulangi untuk pengulangan tetes minyak 2 hingga 5 kali.
\end{enumerate}
\subsubsection*{Pada Kondisi Mengambang}
\begin{itemize}
    \item Arahkan mikrometer okuler secara vertikal dan putar dudukan lensa okuler hingga Anda dapat melihat skala mikrometer dengan jelas.
    \item Pertama-tama, atur sakelar $U$ dan $t$ ke posisi bawah.
    \item Nyalakan tegangan pada kapasitor dengan sakelar $U$ dan sesuaikan dengan potensiometer putar (400-600 V) sehingga tetesan minyak yang dipilih naik dengan kecepatan sekitar 1-2 tanda skala gradasi per detik (yaitu terlihat jatuh saat diamati melalui lensa okuler). Kemudian kurangi tegangan hingga tetesan minyak mengapung.
    \item Matikan tegangan pada kapasitor dengan sakelar $U$.
    \item Begitu tetesan minyak berada pada ketinggian tanda skala gradasi yang dipilih, mulailah pengukuran waktu dengan sakelar $t$.
    Begitu tetesan minyak turun 20 tanda skala gradasi lagi (setara dengan 1 mm), hentikan pengukuran waktu dengan sakelar $t$ dan nyalakan kembali tegangan pada kapasitor dengan sakelar $U$.
    \item Masukkan nilai terukur dari waktu jatuh $t_1$ dan tegangan $U$ dalam tabel dengan stopwatch. Muatan $q$ dapat dihitung dengan persamaan yang tercantum pada bab teori. 
    Ulangi pengukuran untuk tetesan minyak lainnya.
\end{itemize}
\subsubsection*{Pada Kondisi Jatuh-Naik}
\begin{itemize}
    \item Arahkan mikrometer lensa okuler secara vertikal dan putar dudukan lensa okuler hingga Anda dapat melihat skala mikrometer dengan jelas.
    \item Pertama-tama, atur sakelar $U$ dan $t$ ke posisi bawah.
    \item Nyalakan tegangan pada kapasitor dengan sakelar $U$ dan atur dengan potensiometer putar (400-600 V) sehingga tetesan minyak yang dipilih naik dengan kecepatan sekitar 1-2 tanda skala per detik (yaitu terlihat jatuh saat diamati melalui lensa okuler).
    \item Matikan tegangan pada kapasitor dengan sakelar $U$.
    \item Segera setelah tetesan minyak berada pada ketinggian tanda skala yang dipilih, mulailah pengukuran waktu dengan sakelar $t$. 
    \item Begitu tetesan minyak jatuh (yaitu naik seperti yang diamati di lensa mata) dengan 20 tanda skala gradasi lagi (setara dengan 1 mm), hidupkan kembali tegangan pada kapasitor dengan sakelar $U$. Dengan demikian, pengukuran waktu  $t_2$ dimulai secara otomatis. 
    \item Begitu tetesan minyak berada pada ketinggian tanda skala kelulusan pertama lagi, hentikan pengukuran waktu dengan sakelar $t$. 
    \item Masukkan nilai terukur dari waktu jatuh $t_1$, waktu naik $t_2$, dan tegangan $U$ dalam tabel dengan stopwatch. Muatan $q$ dapat dihitung dengan persamaan yang tercantum pada bab teori. 
    \item Ulangi pengukuran untuk tetesan minyak lainnya.
    
\end{itemize}
\subsection{Analisis Data}
                \begin{enumerate}
                \item Untuk Metode \textit{Floating} (Melayang-Layang), lakukan penghitungan  \textbf{masing-masing} besar muatan yang didapatkan dari metode \textit{floating}  (ingat! kita tidak sedang mengukur muatan elektron secara langsung, hanya bisa mengukur muatan dari tetesan minyak) dengan menggunakan persamaan
                 \begin{equation}					
                 \begin{aligned}						
                 \label{mathe}	
                 q_f = \frac{18l\pi\eta v_d}{U_f}\sqrt{\frac{\eta v_d}{2g(\rho_o-\rho_f)}}
                 \end{aligned}
                 \end{equation}
                 
                 dimana $q_f$, dalam Columb, merupakan muatan minyak yang didapatkan saat metode \textit{floating}; $v_d = \frac{y}{t_d}$ merupakan kecepatan terminal saat tetesan minyak itu jatuh dengan $t_d$ adalah waktu untuk menempuh jarak $y$ saat ia jatuh;$\eta$ sebagai viskositas fluida udara sekeliling tetesan minyak ($\eta \approx 1,81 × 10^{-5}$ Ns/m$^2$ ); $\rho_o$ densitas minyak, 875,3 kg/m$^3$; $\rho_f$ densitas fluida udara, 1,29 kg/m$^3$;$l$ merupakan jarak antara 2 plat kapasitor $d=6 × 10^{-3}$m;$U_f$, dalam volt, merupakan tegangan yang diterapkan saat tetesan minyak melayang-layang;  dan $g$ adalah percepatan gravitasi, $9,81$m/s$^2$. 
                 
                \item Untuk metode \textit{Rising-Falling} , gunakanlah persamaan
                 \begin{equation}					
                 \begin{aligned}						
                 \label{mathe}	
                 q_{rf} = \frac{18l\pi\eta (v_d+v_u)}{U_u}\sqrt{\frac{\eta v_d}{2g(\rho_o-\rho_f)}}
                 \end{aligned}
                 \end{equation} 
                 dimana $q_{rf}$, dalam Columb, merupakan muatan minyak yang didapatkan saat metode \textit{Rising and Falling}; $v_u = \frac{y}{t_u}$ merupakan kecepatan terminal saat tetesan minyak itu naik dengan $t_u$ adalah waktu untuk menempuh jarak $y$  saat ia naik; dan $U_u$, dalam volt, merupakan tegangan yang diterapkan saat tetesan minyak melaju naik; 

                 \item Untuk menghitung muatan elementer, lakukan perhitungan selisih dari kemungkinan muatan yang ada. Misalnya, saat sudah dihitung data muatan minyak metode \textit{floating} pertama, $q_{f1}$, kedua, $q_{f2}$, ketiga $q_{f3}$, sampai data ke-30, $q_{f30}$, maka selisihnya adalah $q_{f1}-q_{f2}$,$q_{f1}-q_{f3}$, $q_{f1}-q_{f4}$, ...,dan seterusnya. Demikian juga, dengan cara yang sama, terapkan untuk tetesan minyak yang didapat dari metode \textit{Falling and Rising}.
                 \item Setelah semua selisih itu semua didapatkan, buatlah histogram untuk masing-masing metode (terdapat 2 histogram) dan hitung standar deviasi, $\Delta x$ serta rata-ratanya, $\bar x$. Setelah itu, plot distribusi Gaussiannya (dalam \textit{barchart}) dan jadikan dalam satu grafik. 
                 \item  tentukan juga kuantisasinya melalui perkiraan kelipatan bilangan bulat dari muatan elektron yang telah didapatkan dan susun semua data pada tabel.
                 
            \end{enumerate}
% \begin{figure}[H]
%     \centering
%     \includegraphics[width=0.5\linewidth]{meme.jpg}
%     \caption{...}
%     \label{fig:enter-label}
% \end{figure}
\subsection{Post-Lab}
\begin{enumerate}
    \item Bagaimana distribusi data selisih muatan minyak yang didapatkan, seberapa besar standar deviasinya, berapa rata-ratanya?Jelaskan secara rinci!
    \item Bagaimana data kalian dapat membuktikan dengan adanya kuantisasi muatan dasar?
    \item Apa yang dapat kalian simpulkan dari histogram mengenai selisih antar muatan minyak yang telah kalian analisis?
    \item Jika kecepatan terminal dari tetesan minyak untuk naik lebih tinggi daripada saat jatuh, apa yang sebenarnya terjadi pada tetesan minyak tersebut?
    \item Bandingkan data muatan yang didapatkan dengan nilai sebenarnya, seberapa jauh ralat nisbinya?
    \item Dengan error yang telah didapatkan, jelaskan berbagai penyebabnya dengan rasional dan bagaimana cara menghindarinya atau bagaimana cara memperbaiki sistematika pada eksperimennya? (human error tidak diperkenankan untuk turut serta dibahas)!
\end{enumerate}

