\praktikum[MP5]{Eksperimen Franck-Hertz}
\subsection{Pendahuluan}

\begin{wrapfigure}{r}{0.35\textwidth}
  \centering
  \includegraphics[width=0.4\textwidth]{Gambar/freepik__a-whimsical-cartoon-animation-of-an-electron-perso__32194.jpg} 
\end{wrapfigure}
Pada awal abad ke-20, kita berada di titik di mana pemahaman manusia tentang atom mulai goyah. Model atom Rutherford, yang cukup populer saat itu, tak mampu menjelaskan beberapa fenomena yang tampaknya sederhana, tetapi cukup membingungkan. Salah satunya adalah spektrum emisi atom hidrogen yang tampaknya "terputus-putus," saat elektronnya bermanuver pada tingkat energi satu dengan tingkat energi lainnya. alih-alih berkelanjutan, seperti kita berharap dari sebuah sumber cahaya biasa.\\
Tentu saja, para ilmuwan tak tinggal diam. Dalam upaya untuk memahami permasalahan ini, muncul teori Atom Bohr yang memperkenalkan gagasan bahwa energi dalam atom itu terkuantisasi—artinya, elektron dalam atom tidak bisa memiliki sembarang energi, melainkan hanya energi pada tingkat-tingkat tertentu. Jika model Rutherford adalah gambaran atom yang "begitu saja," maka teori Atom-Bohr ini menggambarkan atom yang jauh lebih rumit dan lebih terstruktur.\\
Di sinilah eksperimen Franck-Hertz mengambil peran. Pada tahun 1914, dua maestro fisika, James Franck dan Gustav Hertz, melahirkan sebuah eksperimen yang mengubah pandangan dunia terhadap struktur atom. Dalam percobaan ini, elektron-elektron dipercepat melalui tabung berisi gas merkuri pada tekanan rendah dan kemudian membombardir atom merkuri. Disinilah energi elektron benar-benar terkuantisasi, Seolah-olah energi ini hanya dapat diakses oleh kunci eksklusif, menunjukkan bahwa eksitasi hanya terjadi pada energi tertentu yang sesuai dengan tingkat energi dalam atom merkuri.

\subsection{Tujuan}
Tujuan dari praktikum ini adalah:
\begin{enumerate}
    \item Merekonstruksi eksperimen Franck-Hertz untuk menghitung energi eksitasi atom.
    \item Menginterpretasikan grafik hasil pengamatan pada Eksperimen Franck-Hertz.
    \item Mengidentifikasi tingkat eksitasi pada atom merkuri.
    \item Menghitung Energi eksitasi.
    
\end{enumerate}
\subsection{Alat dan Bahan}
Peralatan yang digunakan dalam praktimum ini adalah:
\begin{table}[ht]
\centering
\begin{tabular}{ccc}
\toprule
\hline
\textbf{Nama} & \textbf{Jumlah} & \textbf{Detail} \\
\hline
Unit Utama Franck-Hertz & 1 set & - \\
Tabung Franck-Hertz & 1 buah & Tabung vakum berisi uap merkuri \\
Pemanas Tabung & 1 unit & Untuk memanaskan tabung Franck-Hertz \\
Kontroler Suhu & 1 unit & Mengatur suhu pemanas tabung \\
Kabel Penghubung & 1 set & sebagai alat penghubung\\
\hline
\bottomrule
\end{tabular}
\caption{Daftar Alat dan Bahan untuk Eksperimen Franck-Hertz}
\end{table}


\subsection{Pre-Lab}
\begin{enumerate}
    \item Apa yang dimaksud dengan tingkat energi dalam atom? Mengapa elektron dalam atom tidak dapat memiliki sembarang energi?
    \item Mengapa energi eksitasi atom memiliki nilai tertentu dan tidak kontinu? Bagaimana ini terkait dengan model atom Bohr?
    \item Apa yang terjadi pada atom setelah mengalami eksitasi? Jelaskan bagaimana atom kembali ke keadaan dasarnya.
    \item Jika sebuah elektron bermuatan $e$ diberi tegangan tertentu sebesar $V$, Bagaimana hubungan antara tegangan yang diberikan tersebut dengan kecepatan yang dihasilkan pada elektron? Jelaskan alasanmu dengan dekskripsi matematiknya. 
    \item Jika suatu elektron bertumbukan dengan atom tetapi tidak menyebabkan eksitasi, ke mana perginya energi elektron tersebut?
    \item Mengapa energi yang diberikan kepada atom dalam bentuk tumbukan elektron bisa menyebabkan atom berpindah ke tingkat energi yang lebih tinggi?
    \item Bagaimana cara menentukan energi eksitasi atom hanya dari pola perubahan energi elektron yang bertumbukan dengan atom tersebut? 
    \item Jika suatu atom dapat mengalami eksitasi ke beberapa tingkat energi yang berbeda, bagaimana cara mengetahui tingkat energi mana yang dicapai dalam suatu eksperimen?
    % \item Bagaimana spektrum emisi atom dapat digunakan untuk menentukan tingkat energi eksitasi suatu atom? 
    % \item Jelaskan perbedaan antara tumbukan elastik dan tumbukan non-elastik dalam interaksi elektron dengan atom gas.
\end{enumerate}
\subsection{In-Lab}
% \begin{figure}[htbp]
%     \centering
%     \includegraphics[width=0.5\linewidth]{Gambar/Skema_Penjelasan_FH.png}
%     \caption{Skema Alat Percobaan Franck-Hertz}
%     \label{fig:enter-label}
% \end{figure}
\begin{figure}[htbp]
    \centering
    \includegraphics[width=0.8\linewidth]{Gambar/Frannk.png}
    \caption{Diagram Alat Praktikum Percobaan Frank-Hertz}
    \label{fig:enter-label}
\end{figure}
\subsubsection*{Instalasi Alat}
\begin{enumerate}
      \item Nyalakan daya pada Temperature Controller.
      \item Atur suhu furnace ke suhu yang ditentukan.
      \item Tunggu sekitar 15-20 menit hingga suhu furnace mencapai setelan yang ditentukan.
      \item Pantau suhu melalui display temperatur.
      \item Hubungkan kabel sesuai dengan diagram pengkabelan seperti pada Short-circuit G1 dan G2 pada panel depan unit utama.
      \item Jangan menyalakan daya saat sedang melakukan pengkabelan.
      \item Sambungkan kabel sesuai dengan diagram pengkabelan.
      \item Pastikan semua koneksi sudah benar sebelum melanjutkan.
\end{enumerate}
\subsubsection*{Pengukuran Potensial Eksitasi Pertama Atom Merkuri}

\begin{enumerate}
    \item Atur suhu furnace ke 220°C (suhu standar eksitasi pertama).
    \item Tunggu sekitar 15-20 menit hingga suhu furnace mencapai setelan yang ditentukan.
    \item Setelah furnace mencapai suhu yang diinginkan, nyalakan daya unit utama.
    \item Tabung Franck-Hertz dibagi menjadi tiga zona: Zona percepatan (K-G2), Zona tumbukan (G1-G2), Zona koleksi (G2-P).
    \item Atur tegangan $V_f, V_{G1K}$, dan $V_{G2K}$ sesuai dengan data yang tersedia.
    \item Gunakan mode "Auto" untuk mengamati kurva $V_{G2K} - I_P$ secara otomatis.
    \item Atur parameter eksperimen secara manual.
    \item Catat arus pelat IP setiap kenaikan 1.0V dari $V_{G2K} = 0V$ hingga sekitar 60V.
    \item Lakukan penyetelan halus untuk mendapatkan pengukuran tegangan $V_{G2K}$ dengan akurasi lebih tinggi.
    \item Ulangi eksperimen dengan berbagai suhu furnace, tegangan penghambat $V_{G2P}$, dan tegangan filamen $V_F$.
\end{enumerate}

\subsubsection*{Pengukuran Tingkat Eksitasi yang Lebih Tinggi}
\begin{enumerate}
    \item Turunkan suhu furnace dari 220°C ke 100-135°C.
    \item Nyalakan daya unit utama dan atur $V_F$ serta $V_{G2P}$ sesuai kondisi optimal.
    \item Kerapatan atom merkuri dikendalikan melalui suhu furnace.
    \item Pantau kurva arus pelat \(I_P\) terhadap \(U_{KG1}\).
    \item Dengan meningkatnya tegangan akselerasi, jumlah elektron berenergi tinggi bertambah.
    \item Zona G1-G2 menjadi ruang tumbukan dengan resolusi tinggi, memungkinkan puncak spektral yang lebih jelas.
\end{enumerate}
\subsection{Analisis Data}
\begin{enumerate}
  \item Buat tabel pencatatan data.
  \item Plot grafik hubungan antara puncak $V_{G2K}$ dengan $n$ dalam persamaan.
  \begin{equation}
      nV_0+V_c = V_{G2K}
  \end{equation}
  dengan $V_c$ adalah potensial kontak
  \item Lakukan hal serupa baik pada Langkah Eksitasi pertama maupun eksitasi yang lebih tinggi
  \item Untuk eksitasi yang lebih tinggi, gunakan tabel berikut untuk menunjukkan Tingkat keadaan eksitasi-nya
  \item Kemudian cari nilai energi eksitasi dengan menggunakan persamaan 
  \begin{equation}
      E =  q\cdot (V_{i+1}-V_i)
  \end{equation}
  \item Hitung ketidakpastian pengukuran.
\end{enumerate}
\begin{table}[ht]
\centering
\begin{tabular}{ccc}
\toprule
\hline
\textbf{Puncak $V_{G2K}$ (V)} & \textbf{Energi Eksitasi (eV)} & \textbf{Tingkat Eksitasi} \\
\hline
4.9V & 4.9 eV & $6^3 P_1$ (eksitasi pertama) \\
6.7V & 6.7 eV & $6^1 P_1$ (eksitasi singlet P) \\
7.73V & 7.73 eV & $6^3 P_2$ (eksitasi triplet lebih tinggi) \\
10.4V & 10.4 eV & $7^1 S_0$ (tingkat ionisasi) \\
\hline
\bottomrule
\end{tabular}
\caption{Tabel Energi Eksitasi dan Tingkat Eksitasi}
\end{table}

\subsection{Post-Lab}
\begin{enumerate}
    \item Mengapa hanya kelipatan tingkat energi eksitasi dasar saja yang dapat menyerap energi dalam atom?
    \item Bagaimana tumbukan elastis dan inelastis terjadi?
    \item Bagaimana tumbukan elastis dan inelastis dapat menentukan perhitungan tingkat energi dalam atom?
    \item Apa maksud dari pembacaan arus dalam tabung? dan Jelaskan juga hasil interpretasi Hasil arus pada Tabung Frank-Hertz ini
    \item Jika gas merkuri diganti dengan Neon, apa yang akan terjadi pada eksperimen Franck-Hertz?
    \item Bagaimana percobaan tersebut dapat berhubungan dengan teori atom Bohr?
    \item Apa saja error yang didapat dari percobaan yang Anda lakukan?
\end{enumerate}

