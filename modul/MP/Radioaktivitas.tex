\praktikum[MP4]{Radioaktivitas dan \\ Pengukuran Waktu Paruh}
\subsection{Pendahuluan}

\begin{wrapfigure}{r}{0.35\textwidth}
  \centering
  \includegraphics[width=0.4\textwidth]{Gambar/radioaktivitas_stoples.png} 
\end{wrapfigure}
Radioaktivitas adalah wujud keabadian dalam bentuk yang lebih gelap dan juga cukup menegangkan. Inti atom yang tidak stabil, yang mungkin tengah merasa nyaman dengan keadaannya, tiba-tiba melepaskan partikel atau radiasi elektromagnetik ke sekitarnya. Radiasi ini bisa berupa partikel alfa, beta, atau bahkan sinar gamma yang, meskipun tidak tampak, bisa berbahaya bagi tubuh kita.

Tapi tenang saja, sebagai manusia yang cukup rentan sama radiasi ini, tentu kita perlu perlindungan. Lalu, gimana caranya kita melindungi diri dari ancaman radiasi yang tiba-tiba datang begitu saja? Cukup menggunakan bahan pelindung yang didesain untuk menyerap atau mengurangi kekuatan radiasi tersebut. Tapi, tentu saja, nggak semua bahan pelindung itu sama. Efektivitasnya tergantung pada jenis radiasi yang datang, energi yang dibawa, dan, tentu saja, ketebalan serta kepadatan bahan pelindung yang digunakan. Oh, dan jangan lupa, jarak. Makin jauh dari sumber radiasi, makin aman—itu hukum dasar yang bisa kita percaya.

\subsection{Tujuan}
Tujuan dari eksperimen ini adalah:
\begin{enumerate}
    \item Menyelidiki fenomena radioaktivitas,
    \item Memahami bagaimana ketidakstabilan nuklir menyebabkan radioaktivitas,
    \item Memahami konsep waktu paruh dan konstanta peluruhan,
    \item Mengukur waktu paruh suatu bahan radioaktif,
    \item Memahami bagaimana intensitas radiasi dipengaruhi oleh jarak,
    \item Memahami bagaimana perlindungan terhadap radiasi bekerja,
    \item Memahami prosedur keselamatan dalam menangani radioisotop dengan menggunakan bahan pelindung.
\end{enumerate}

\subsection{Peralatan}
Peralatan yang digunakan dalam praktimum ini adalah:
\begin{table}[htbp]
    \centering
    \begin{tabular}{lcc}
        \toprule
        \midrule
        \textbf{Nama} & \textbf{Jumlah} & \textbf{Detail} \\
        \midrule
        Dudukan statis & 2 & - \\
        Mobile CASSY & 1 & - \\
        Laptop & 1 & -\\
        Tabung GM (Geiger-Müller) & 1 (Adaptor disertakan) & - \\
        Peralatan elusi & - & -  \\
        Sumber radioaktif & 5 (Cs-137, Am-241, Co-60, Kr-54, Sr-90) &-\\
        Sarung tangan lateks & 2 & - \\
        Kertas, plastik, dan logam & Masing-masing 1 & -\\
        \midrule
        \bottomrule
    \end{tabular}
\end{table}

\subsection{Pre-Lab}
\begin{enumerate}
    \item Bagaimana radioaktivitas terjadi? Apa yang terjadi dalam inti atom radioaktif?
    \item Jelaskan konsep waktu paruh dalam konteks peluruhan radioaktif.
    \item Apa hubungan antara konstanta peluruhan dan waktu paruh isotop radioaktif?
    \item Jelaskan konsep kesetimbangan radioaktif dan bagaimana hal itu mempengaruhi aktivitas sampel radioaktif seiring waktu.
    \item Uraikan skema peluruhan dari Cs-137.
    \item Mengapa larutan elusi yang digunakan adalah larutan NaCl?
    \item Sebutkan tiga jenis radiasi utama yang dipancarkan selama peluruhan radioaktif dan karakteristiknya.
    \item Gambarkan ilustrasi yang menunjukkan hubungan antara intensitas radiasi dan jarak.
    \item Bagaimana bahan tertentu dapat menghentikan radiasi?
    \item Bagaimana perlindungan terhadap radiasi diterapkan dalam penanganan bahan radioaktif?
\end{enumerate}

\subsection{In-Lab}
\begin{figure}
    \centering
    \includegraphics[width=0.8\linewidth]{Gambar/Radioaktivitas1.png}
    \caption{Skema Alat untuk Elusi Cs-137}
    \label{fig:enter-label}
\end{figure}
\subsubsection*{Elusi Cs-137}
\begin{enumerate}
    \item set up alat praktikum menurut gambar 6.
    \item Gunakan selang plastik untuk memasukkan 2-3 mL larutan elusi ke dalam jarum suntik.
    \item Dengan menggunakan sarung tangan lateks, buka penutup pelindung generator Cs-137.
    \item Pasang jarum suntik ke generator dengan memutarnya ke tempatnya.
    \item Dengan hati-hati, dorong larutan elusi ke dalam generator sambil menempatkan tabung uji di bawahnya.
    \item Proses elusi harus berlangsung selama 10-20 detik.
\end{enumerate}

\subsubsection*{Pengukuran Waktu Paruh}
\begin{figure}
    \centering
    \includegraphics[width=0.8\linewidth]{Gambar/Radioaktivitas2.png}
    \caption{Skema Alat untuk Pengukuran Waktu Paruh}
    \label{fig:enter-label}
\end{figure}
\begin{enumerate}
        \item set up alat praktikum menurut gambar 7
    \item Hubungkan Mobile CASSY ke GM-Adaptor dan Tabung GM.
    \item Atur waktu pengukuran di Mobile CASSY menjadi manual dan mulai pengukuran aktivitas radioaktif (dalam hitungan per detik).
    \item Hentikan pengukuran jika aktivitas radioaktif mendekati 1 hitungan/detik.
    \item Simpan hasil pengukuran sebagai file teks.
\end{enumerate}

\subsubsection*{Pengukuran Intensitas Radiasi dan Pelindung}
\begin{enumerate}
    \item set up alat praktikum menurut gambar 8.
    \item Posisikan Tabung GM sejajar dengan sumber radiasi.
    \item Ukur intensitas radiasi pada berbagai jarak dari 1 cm hingga 10 cm.
    \item Catat intensitas radiasi latar belakang di dalam ruangan.
    \item Uji efek bahan pelindung (kertas, plastik, dan logam) terhadap intensitas radiasi.
    \item Rekam hasil pengukuran untuk setiap variasi.
\end{enumerate}
\begin{figure}[htbp]
    \centering
    \includegraphics[width=0.8\linewidth]{Gambar/Radioaktivitas3.png}
    \caption{Skema Alat untuk Pengukuran Intensitas Radiasi dan Pelindung}
    \label{fig:enter-label}
\end{figure}
\warningbox{Setelah melakukan pratkikum, pastikan sumber radiasi disimpan dalam wadah yang terlindungi dan rapat}
\subsection{Analisis Data}
\begin{enumerate}
    \item Aktivitas zat radioaktif pada waktu $t$, $A_t$, diberikan oleh persamaan:
    \begin{equation}
        A_t = A_0 e^{-\lambda t}
    \end{equation}
    \item Sesuaikan data dengan regresi eksponensial atau lakukan linearisasi terlebih dahulu.
    \item Jika konstanta peluruhan $\lambda$ diketahui, maka waktu paruh dapat dihitung dengan:
    \begin{equation}
        T_{1/2} = \frac{\ln(2)}{\lambda}
    \end{equation}
    \item Buat grafik intensitas radiasi sebagai fungsi jarak.
\end{enumerate}

\subsection{Post-Lab}
\begin{enumerate}
    \item Apa yang dapat disimpulkan dari aktivitas radioaktif sebagai fungsi waktu?
    \item Bahan apa yang bersifat radioaktif dalam eksperimen ini?
    \item Dari bahan yang diuji, mana yang memiliki waktu paruh tertentu?
    \item Apa arti fisik dari konstanta peluruhan?
    \item Mengapa kurva yang dihasilkan tidak berbentuk eksponensial sempurna?
    \item Mengapa radiasi berbanding terbalik dengan jarak?
    \item Bagaimana bahan pelindung dapat menghentikan radiasi?
    \item Identifikasi jenis radiasi dari masing-masing sumber radioaktif berdasarkan data pengukuran.
\end{enumerate}